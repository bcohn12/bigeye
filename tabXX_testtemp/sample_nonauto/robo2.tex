% Use only LaTeX2e, calling the article.cls class and 12-point type.
%\typeout{}\typeout{If latex fails to find aiaa-tc, read the README file!}
\documentclass[onecolumn]{IEEEtran}

% Users of the {thebibliography} environment or BibTeX should use the
% scicite.sty package, downloadable from *Science* at
% www.sciencemag.org/about/authors/prep/TeX_help/ .
% This package should properly format in-text
% reference calls and reference-list numbers.

%\usepackage{scicite}

% Use times if you have the font installed; otherwise, comment out the
% following line.

\usepackage{times}

% The preamble here sets up a lot of new/revised commands and
% environments.  It's annoying, but please do *not* try to strip these
% out into a separate .sty file (which could lead to the loss of some
% information when we convert the file to other formats).  Instead, keep
% them in the preamble of your main LaTeX source file.

%%%
\newcommand{\real}{\mathbb{R}}
\newcommand{\bfb}[1]{\mbox{\boldmath $ #1 $ }}
\newcommand{\R}{{\ensuremath{{\mathbb{R}}}}}
\providecommand{\norm}[1]{\lVert#1\rVert}


\renewcommand{\vec}[1]{\bm{\mathrm{#1}}}

\def \ell{l}
\def \cD{\mathcal{D}}
\def \cDb{\cD_{\textrm{b}}}
\def \cDf{\cD_{\textrm{f}}}
\def \ellmax{\ell_{\textrm{max}}}
\def \nref{n_{\textnormal{ref}}}
\def \hell{h^{\ell}}
\def \hellminus{h^{\ell-1}}
\def \normomega{\| \vec{\omega} \|}

\usepackage[table]{xcolor}
\usepackage{psfrag}
\usepackage{amssymb}
\usepackage{amsmath}
\usepackage{latexsym}
\usepackage{setspace}
\usepackage{wrapfig}%   wrap figures/tables in text (i.e., Di Vinci style)
%\usepackage{epsfig}
%\usepackage{algorithm}
%\usepackage{algorithmic}
\usepackage{graphicx}
%\usepackage{svninfo}
%\usepackage{fancyhdr}
%\usepackage{subfigure}
%\usepackage[usenames]{color}
%\usepackage[top=1.0in, bottom=0.75in, left=0.75in, right=0.75in]{geometry}
%\usepackage{lineno}
\usepackage{color}
%\usepackage{cite}
\usepackage{bm}
\usepackage[]{natbib}
\usepackage{url}

%\usepackage{datetime}
\usepackage{verbatim}

%\setlength\linenumbersep{3pt}

%\settimeformat{ampmtime}
\DeclareGraphicsExtensions{.jpg,.pdf,.mps,.png} %For pdftex

\graphicspath{{../figs/ai_tweaked/}}

\newcommand{\mus}{$\mu$S~$\cdot$~cm$^{-1}$}
\newcommand{\muv}{$\mu$V~$\cdot$~cm$^{-1}$}
\newcommand{\mvcm}{mV~$\cdot$~cm$^{-1}$}
\newcommand{\cus}{cm~$\cdot$~s$^{-1}$}
\newcommand{\acm}{cm~$\cdot$~s$^{-2}$}
\newcommand{\pmm}{mm$^{-2}$}
\newcommand{\mml}{mmol~$\cdot$ l$^{-1}$}
\newcommand{\daph}{{\it Daphnia}}
\newcommand{\albif}{{\it A.~albifrons}}
\newcommand{\lepto}{{\it A.~leptorhynchus}}
\newcommand{\apt}{{\it Apteronotus}}
\newcommand{\IDNcom}[1] {\textit{\color{red} #1 -IDN}}
\newcommand{\RBcom}[1] {\textit{\color{blue} #1 - Rahul Bale}}
%%%

% The following parameters seem to provide a reasonable page setup.

\topmargin 0.0cm
\oddsidemargin 0.2cm
\textwidth 16cm
\textheight 21cm
\footskip 1.0cm

\doublespacing
\usepackage{lineno}
\linenumbers

% If your reference list includes text notes as well as references,
% include the following line; otherwise, comment it out.

\renewcommand\refname{References and Notes}
\begin{document}


\title{Undulating Fins Produce Off-Axis Thrust and Flow Structures}


%\author{\vspace*{0.5in}}
\author{{Izaak D. Neveln$^{1}$, Rahul Bale$^{2}$, Amneet Pal Singh Bhalla$^{2}$, Oscar M. Curet$^{2,4}$, \\
Neelesh A. Patankar$^{2}$, Malcolm A. MacIver$^{1,2,3,*}$}\\
\normalsize{$^1$Department of Biomedical Engineering, Northwestern University, Evanston, IL, USA}\\
\normalsize{$^2$Department of Mechanical Engineering, Northwestern University, Evanston, IL, USA;}\\
\normalsize{$^3$ Department of Neurobiology and Physiology, Northwestern University, Evanston, IL, USA}\\
\normalsize{$^4$ Present address: Department of Ocean and Mechanical Engineering, Florida Atlantic University, Boca Raton, FL, USA}\\
\normalsize{$^*$ Author for correspondence (e-mail: maciver@northwestern.edu)}}
% Include the date command, but leave its argument blank.

%\date{}



%%%%%%%%%%%%%%%%% END OF PREAMBLE %%%%%%%%%%%%%%%%



%\baselineskip24pt  % Double-space the manuscript.

% Make the title.

\maketitle


% Place your abstract within the special {sciabstract} environment.
\begin{abstract}
\noindent
While wake structures of many forms of swimming and flying are well
characterized, the wake generated by a freely-swimming undulating fin has
not yet been analyzed. These elongated fins allow fish to achieve enhanced
agility exemplified by the forward, backward, and vertical swimming
capabilities of knifefish
and also have potential applications in the design of more maneuverable
underwater vehicles. We present the flow structure of an undulating
robotic fin model using particle image velocimetry to measure fluid
velocity fields in the wake. We supplement the experimental robotic work
with high-fidelity computational fluid dynamics, simulating the
hydrodynamics of both a virtual fish whose fin kinematics and fin plus body
morphology are measured from a freely-swimming knifefish as well as a virtual rendering of our robot. Our results
indicate a series of linked vortex tubes is shed off the long edge of the fin as
the undulatory wave travels lengthwise along the fin. A jet at an oblique
angle to the fin is associated with the successive vortex tubes,
propelling the fish forward. The vortex structure bears similarity to the
linked vortex ring structure trailing the oscillating caudal fin of a
carangiform swimmer, though the vortex rings are distorted due to the
undulatory kinematics of the elongated fin.

\end{abstract}


\textbf{Key words:} fish locomotion, undulating fin propulsion, knifefish, bio-mimetic, bio-robotics, digital particle image velocimetry (DPIV)

\section{Introduction}
\noindent
A large variety of fish undulate elongated fins to generate thrust.
Knifefish undulate either an elongated dorsal fin, as is the case of the
African knifefish \emph{Gymnarchus niloticus}, or anal fin, as is the
case of species in the order Gymnotiforms and the family Notopteridae.
Triggerfish of the family Balastidae undulate both dorsal and ventral
fins. Also, many species such as cuttlefish, skates, and rays undulate
elongated fins along their lateral margins. Undulatory motion, in contrast to oscillatory
motion, implies that these fins produce a wave that travels along the fin
with at least one full undulation present along the fin. These fish propel
primarily by median and/or paired fin (MPF) propulsion, in contrast to
body and/or caudal fin (BCF) propulsion where the fish propels by the
actuation of the whole body, part of the body or only the caudal fin [for review:
\citep*{Sfak99a}].

The biological model system used for this research is the elongated anal fin (hereafter termed
the ribbon fin) of the weakly electric knifefish, such as the fin present on black ghost
knifefish \emph{Apteronotus albifrons} in Fig. \ref{fig:fish_robot}A. These
knifefish are important model systems for studying sensory neurobiology
\citep*{Turn99a, Krah13a} and sensory motor
integration \citep*{Snyd07a,MacI10a,Cowa07a}. Furthermore, research on the unique
biomechanics and sensory modality of electric knifefish have inspired
numerous underwater robots capable of agile movement and omnidirectional
close-range sensing detailed in a review by Neveln et al. \citep*{Neve13a}.

Blake pioneered the study of ribbon fin propulsion in his study
of knifefish \citep*{Blak83a} in which he described the
kinematics of forward swimming fish.  Subsequent work has quantified body 
kinematics \citep{MacI01a} as well as fin kinematics \citep{Ruiz13a}. Lighthill and Blake first modeled 
the hydrodynamics of the undulating ribbon
fin using elongated-body theory \citep*{Ligh90a}. Work in our lab has examined the
hydrodynamics of a non-translating undulating fin \citep*{Shir08a}, as well as
measured the forces and free-swimming velocities generated by a biomimetic
robotic ribbon fin shown in Fig. \ref{fig:fish_robot}B \citep*{Cure11a}. Results
show that a single traveling wave
along the fin is capable of producing both surge and heave components of
thrust (see Fig. \ref{fig:fish_robot}C for axes definitions). Knifefish
can also partition their ribbon fin into two counter-propagating waves
that originate on either end of the fin 
\citep*{Cure11b}. Knifefish control the position of the point where the two
waves meet along the fin
to hover in still water and swim at slow speeds \citep*{Ruiz13a, Sefa12a}.
These counter-propagating waves on a single ribbon fin can be used to vary the
proportion of surge and
heave components of thrust \citep*{Cure11b, Sefa12a}.

Because of the unique properties of the propulsive mechanisms of the
ribbon fin, weakly electric knifefish exhibit high maneuverability, including backward, upward, and rolling motions. 
Even for time periods as short as a small multiple of the sensorimotor neuronal delay time (\textasciitilde 100~ms),
they can maneuver to any location within their omnidirectional sensing range for small
prey \citep*{Snyd07a}. This tight coupling of sensory, neuronal, and
mechanical systems presents an interesting case study in the co-evolution
of these complex, interdependent systems and the inherent trade-offs that
occur when trying to maximize a relevant objection function such as the 
ratio of energy intake to expended energy \citep*{MacI10a}. A
better understanding of the intricate mechanical properties of the ribbon
fin is essential for further decoding the interdependency of sensory, neuronal,
and mechanical systems in this model system.

This study aims to further our understanding of hydrodynamic principles
underlying cruising ribbon fin swimmers through experimentation on a robotic model using 
digital particle image velocimetry as well as high-fidelity computational 
fluid dynamics of a virtual model of our robot as well as an accurate model with measured body and fin 
morphology combined with measured ribbon fin kinematics. We first
describe the flow features present in the wake which are responsible for
generating thrust along both the surge and heave axes of the fin. We then
show the consistency of this flow structure across a range ribbon fin 
sizes and kinematics with Reynolds numbers spanning over one order of magnitude. Finally,
we introduce a schematic of the ribbon fin wake vortex structure and
compare MPF undulatory swimming to well known BCF swimming mechanics under
a framework for comparing flow structures using nondimensional parameters
such as wave efficiency and the Strouhal number.

\section{Materials and Methods}
\subsection{The Robotic Knifefish}

To study the flow structures of ribbon fin propulsion, we used a
mechanical fin model that mimicked the undulatory kinematics of the
ribbon fin. The mechanical fin was a slightly modified version of the
knifefish-inspired `Ghostbot' (Fig. \ref{fig:fish_robot}B) used in a
number of previous studies \citep*{Cure11a,Cure11b,Sefa12a}. The fin
remained 32.60~cm long ($L_\text{fin}$), but was modified to be 5.00~cm
deep ($h_\text{fin}$) rather than 3.37~cm in prior studies to generate a higher amount of thrust.  
The Ghostbot is roughly three times the size of an adult black 
ghost like the one shown in Fig. \ref{fig:fish_robot}A. The fin was
actuated by 32 stainless steel rays, emulating the bony rays of the
knifefish ribbon fin, each driven independently by a servo motor
\citep*{Cure11a}. The fin membrane connecting these rays was
lycra fabric, which had an elastic modulus of $\approx 0.2$~MPa,
similar to the collagenous membrane between fish fin rays \citep*{Laud06b}.
The motors and other hardware were housed in a submersible cylindrical
body (6.35 cm diameter) with rounded-conical caps at both ends. The total
length of the robot was 60 cm.

\subsection{Flow tunnel and particle image velocimetry (PIV)}

The robot was submerged in a variable-speed flow tunnel and supported from
above on frictionless air bearings (New Way Air Bearings, Aston, PA, USA),
as shown in Fig \ref{fig:exp_setup}A. The working section of the tunnel
was 100~cm long, 33~cm wide, and 32~cm deep. The air bearings permit
movement along the longitudinal direction of the flow tunnel; however, the
platform can also be fixed to mechanical ground through a force transducer
(LSB200, Futek, Irvine, CA, USA), so that longitudinal forces can be
measured, similar to the methods used in \citep*{Cure11a}.

A digital PIV system was built to illuminate 2D planes of fluid in the water
tunnel. The PIV system was modeled after \citep*{Schl07a}. Briefly, the beam
from a 2 watt laser (Verdi G2, Coherent Inc., Santa Clara, CA, USA) was
directed to an oscillating galvanometer mirror system (6215HM40B,
Cambridge Technology Inc., Bedford, MA, USA). This oscillating mirror was located at
the focal point of a custom designed parabolic mirror, so that successive instances of the
beam reflected off the parabolic mirror remained parallel, creating a strobed laser-light sheet.
This light sheet was then directed into the water, illuminating silver
coated glass spheres immersed in the fluid (44 micron beads, Potter
Industries, Valley Forge, PA, USA). High speed video (FastCam 1024P PCI,
Photron, San Diego, CA, USA) of the illuminated plane was recorded at the
same rate that the laser beam was scanned, so that each frame recorded one
transversal of the laser beam. In all experiments, the image capture rate was 500
frames per second.  The width of the laser sheet was approximately 30~cm.

PIV data was analyzed using a commercial software package (DaVis, LaVision
GMBH., G\"{o}ttingen, Germany). Analysis consisted of a calibration using
an evenly spaced grid, video data importation, and cross-correlation for
velocity vector calculation. Two passes were performed for each pair of
images, decreasing from an interrogation window of 64 by 64 pixels with
50\% overlap on the first pass to 32 by 32 pixels with 50\% overlap on the
second pass. Subsequent analysis of the velocity vectors was performed in
MATLAB (The MathWorks, Natick, MA, USA), as described below.

\subsection{Experimental Procedure}

Previously, \citet*{Cure11a} investigated the fixed-swimming forces and
free-swimming velocities generated by the fin for a range of kinematic
parameters describing an idealized traveling sinusoidal wave. The wave
used in the study prescribed the trajectories of fin ray angles relative
to the vertical line, called the angular amplitude ($\theta$), given by
the following equation:
\begin{equation} \theta_\text{rays}(x_\text{fin},t)=\theta_\text{max}
\sin \left( 2 \pi \left( \frac{x_\text{fin}}{\lambda}+ ft \right) \right) \label{eq:theta_waves}
\end{equation}
where $x$ is the surge axis along the fin from Fig. \ref{fig:fish_robot}C.
 The kinematic parameters are schematized in Fig. \ref{fig:fish_robot}C and
include maximum angular amplitude $\theta_\text{max}$, frequency of ray
oscillation $f$, and wavelength $\lambda$ (usually reported as
$\frac{L_\text{fin}}{\lambda}$ which we refer to as number of undulations). For this study, a single set of kinematic
parameters was chosen to be the nominal traveling wave ($\theta_\text{max} =
30^\circ$, $f = 3$~Hz, and number of undulations = 2). All PIV data
were collected with these fin kinematic parameters.

While \citet*{Cure11a} only measured free-swimming velocities in the surge
direction,
they noticed that the fin produces both surge and heave force components
for single-traveling wave kinematics. Therefore, for a better
measure of free-swimming performance, the fin must be pitched at the angle
where the overall direction of thrust is aligned with the direction of
flow. To estimate this angle, we performed a series of experiments over a
range of pitch angles from $9^\circ$--$19^\circ$ in which we expected to find 
the correct angle. For each angle, we fixed the
robot to mechanical ground through a load cell to measure longitudinal
force.  Our setup did not allow for measurement of the vertical force. 
We then varied the speed of the flow around the swimming velocity
expected for the nominal kinematic fin parameters while measuring force,
where negative force indicated the robot swimming below its free-swimming
speed and positive force indicated swimming above its free-swimming speed.
The result was a linear relationship between flow speed and force with a
zero crossing at the actual free-swimming speed where thrust and drag
cancel. We also collected PIV data of the midsagittal plane, which
indicated the direction of the wake generated by the fin. We found that at
a pitch angle of $13^\circ$, the swimming speed was maximized and the wake
was aligned with the flow. Thus, for collection of all subsequent PIV data, we
pitched the fin at $13^\circ$ relative to the direction of flow and set
the flow speed to the measured free swimming speed at that pitch angle.
In these experiments, the deepest point of the fin was always twice 
the boundary layer thickness from the bottom of the tank.

The PIV planes chosen to investigate the wake generated by the fin are
shown in Fig. \ref{fig:exp_setup}B. The PIV system is designed to allow
for sagittal and horizontal planes parallel with the flow direction and
transverse planes perpendicular to the flow direction. 
Midsagittal plane (M) and horizontal plane (H) are composites of multiple
PIV data sets collected separately at different areas of the plane due to the limited width of the laser sheet
(original cross-correlation on the images was performed separately as
well). PIV sets from a single plane were spatially and temporally aligned. Data were linearly interpolated onto a
uniform grid for display, with a vector spacing greater than the outcome
of the original cross-correlation of the PIV data. A five frame moving
average was used to temporally filter the data, an important step for the
transverse planes (T1-T5) because of the noise in the data due to the fact
that particles generally only stayed within the laser light sheet for a small number of frames. Lastly, the
velocity vector fields were phase averaged over five full cycles ($T_\text{cycle}
= {f}^{-1} = 333$~ms).

\subsection{Numerical simulations}

In order to relate the flow structures of our robotic model back to that
of the actual fish, we used fluid dynamics simulations to step from a
computational model based on the body shape \citep*{MacI00a}, fin shape,
and fin kinematics \citep*{Ruiz13a} of \albif ~to a computational model of
the Ghostbot fin shape and kinematics. The first simulation, FishSim1, most
closely matches the kinematics measured for the 14.9~cm/s case from
\citet*{Ruiz13a}. A sinusoidal traveling wave was implemented on the
measured fin morphology with the time averaged kinematic parameters
reported in \citet*{Ruiz13a} (see Table \ref{tab:exp_sets} for details).  
The amplitude profile of FishSim1 was based on the maximum excursion of each ray of the actual fish kinematics.
In FishSim2, the morphology, frequency, and number of undulations remained
the same as FishSim1, and only the amplitude profile was changed to a
constant 30$^\circ$ as indicated by the blue shaded entries in Table
\ref{tab:exp_sets}. RoboSim maintained the same constant amplitude profile
of FishSim2, but the robot body and fin morphology were modeled. Also, the
frequency and number of undulations were adjusted to match the actual robot
case, indicated by the red shaded entries in Table \ref{tab:exp_sets}.
However, the fin wave speeds ($f\lambda$) of the RoboSim and the FishSims were similar (48
cm/s and 40 cm/s respectively). Lastly, we also include FinSim, which
simulates an undulating rectangular fin on the scale of the knifefish fin
without a body. The details of the computational method are described
below.

Our computational method uses the constraint-based immersed body (cIB) 
method ~\citep*{Bhal13,Shar05a,Pata00a,Pata01a,Pata05a,Glow99a,Shir09a,Cure10a} 
which is a variant of regular immersed body (IB) method of Peskin~\citep*{Pesk2002}. 
The cIB method solves the combined momentum equation of the fluid and immersed body 
(incompressible Navier-Stokes equation) while imposing the constraint of the prescribed 
motion of the immersed structure in a computationally efficient manner. Specifically, the 
combined momentum equation is solved in the computational domain $\cD$ represented on 
a Cartesian grid in the Eulerian frame of reference. The immersed body is modeled as 
a Lagrangian collection of immersed particles that are free to move on the background 
Eulerian grid. The immersed particles do not require any connectivity information and thus
no explicit meshing is needed in the body domain, as is generally needed in the finite element 
type (FEM) methods. The presence of the immersed body is modeled via an additional body force in the 
incompressible Navier-Stokes equation that is present only inside the volume region 
physically occupied by the immersed body at any given time-instant. The IB method and its variants 
like the cIB method, permit the usage of fast Cartesian grid solvers which are otherwise difficult to use
with the unstructured grids employed within the FEM framework. 

To futher reduce the computational cost 
of resolving fine flow features near the body of the fish and the robot, we solve the incompressible 
Navier-Stokes equation on block-structured locally refined Cartesian grid as detailed by Griffith
~\citep*{Grif09b,Grif12a,Grif05a,Grif05b,Grif07a,Grif09b}. The locally refined Cartesian 
grid consists of hierarchy of grid levels which are denoted by $\ell=0,1,\ldots,\ellmax$. 
Level $\ell = 0$ denotes the coarsest level in the grid hierarchy and  $\ellmax \ge 0$ denotes
the finest level.  Each level of the hierarchical grid is composed of
one or more rectangular Cartesian grid patches on which the combined
momentum equation is discretized.  We denote the Cartesian grid spacing on level $\ell$ of 
the locally refined grid by $\hell$. This grid spacing is related
to the grid spacing on next coarser level $\ell-1$ by an integer refinement 
ratio $\nref^{\ell-1}$, so that $\hell = \frac{1}{\nref^{\ell-1}} \hellminus$.  
The immersed structure is placed on the finest grid level so that the thin 
boundary layers are accurately resolved. We also tag regions of $\cD$ where 
the magnitude of vorticity vector, $\normomega$, reaches a certain threshold value. 
This helps us to deploy fine grid structures to accurately capture the vortex
structures shed from the immersed interfaces without requiring finer resolution
in other parts of $\cD$. The details of the numerical method 
and its validation for various benchmark problems can be found in~\citep*{Bhal13}.

The outlined numerical scheme is implemented in IBAMR \citep*{IBAMR-web-page},
an open-source C++ library that provides support for fluid-structure 
interaction models that are based on the IB method. IBAMR uses SAMRAI
\citep*{samrai-web-page,Horn02a,Horn06a}, PETSc
\citep*{petsc-web-page,petsc-user-ref,petsc-efficient}, \emph{hypre}
\citep*{hypre-web-page,Falg02a}, and \texttt{libMesh}
\citep*{libMesh-web-page,kirk06a} libraries, among others for its
functionality.      

The simulations were carried out in a three-dimensional rectangular computational domain $\mathcal{D}$. 
In all simulations, zero-velocity boundary conditions were imposed on the domain boundaries, except for the
simulations done for the FinSim case. For the FinSim case, periodic boundary conditions were imposed on the domain in 
the rostrocaudal direction, and zero-velocity boundary conditions were used for the remaining boundaries. Details of 
the domain size, refinement ratio, number of grid levels and coarsest grid size, and vorticity threshold used for 
tagging high-gradient vortex regions are reported in Table~\ref{tab:sim_par}.


\subsection{Computing Environment}
The numerical simulations were carried at the Northwestern University Quest supercomputing facility. The cluster is made up of 502 nodes 
each consisting of eight 64-bit, 2.26Ghz Intel Nehalem E5520 processors with 48Gb of memory for each node. 


\section{Results}

\subsection{Free-swimming velocities}

The free-swimming pitched robot achieved a swimming speed of
26~cm/s (0.8 $L_\text{fin}$/s) for the kinematic parameters
tested ($f = 3$~Hz, number of undulations $=2$, $\theta_\text{max} = 30^\circ$). 
Simulation FishSim1, which used the mean kinematic parameters of an actual
free-swimming \albif ~(14.9~cm/s data, from \citet*{Ruiz13a}), obtained a steady state swimming speed of
12~cm/s (1.33 $L_\text{fin}$/s). This speed underestimates the actual
swimming speed of the fish by approximately 20\%. Simulation
FishSim2, which is the same morphology and kinematics as FishSim1
except that the amplitude envelope was constant at $30^\circ$,
obtained a steady state swimming speed of 17.7~cm/s (1.97
$L_\text{fin}$/s).  GhostbotSim obtained a
steady state swimming speed of 26~cm/s (0.80 $L_\text{fin}$/s).
The swimming speed matched 
the experimental robotic results very closely, providing validation of the 
simulation at the higher Reynolds number of the robot (Re = 80000) over the fish (Re = 13000).  
Finally, FinSim swam 9~cm/s (0.89 $L_\text{fin}$/s).
These results, as well as important kinematic parameters and
non-dimensional numbers such as wave efficiency and Strouhal
number, are reported in Table \ref{tab:exp_sets}.

\subsection{Wake patterns from PIV planes}

\subsubsection{Midsagittal plane}

Fig. \ref{fig:midsag} shows both snapshots the velocity
vector field with corresponding vorticity as well as the average
jet generated by the fin in the midsagittal plane M. In a
particular time instant (Fig. \ref{fig:midsag}A), velocity
vectors (which have the incoming flow velocity subtracted) have
varying directions along the length of the fin, alternating from
more aligned in the surge direction to more aligned in the heave
direction. A line of clockwise rotating vortices are
shed at an angle to the fin similar to the pitch angle of the
fin. Behind the fin, there are areas of vorticity with
alternating sign.  Fig. \ref{fig:midsag}B shows the 
phase-averaged velocity vector field at the same instant 
in the cycle as Fig. \ref{fig:midsag}A. Electronic Supplementary Movie 1 (SM1) shows
the time evolution of the velocity vector field with associated
vorticity over one cycle.

On average, the jet generated by the fin is aligned with the
flow direction, as expected if thrust is aligned with swimming
direction (Fig. \ref{fig:midsag}C). This jet is strongest and deepest just downstream of
the caudal end of the fin. The depth of the jet increases
approximately linearly from the rostral edge of the fin to the
caudal edge, and tapers off further downstream. There is a
smaller secondary jet just downstream of the fin aligned along
the surge axis of the fin.

\subsubsection{Horizontal plane}

Fig. \ref{fig:horiz} shows four snapshots of the phase averaged velocity vector
field and associated vorticity for the horizontal plane H. These
snapshots split the fin cycle into quarters. Velocity vectors
(which have the incoming flow subtracted) in the area of
the plane transecting the fin are oriented largely laterally.
Just downstream, vortices are shed off of the fin into the
plane. These vortices alternate in rotational direction and are
shed twice per fin cycle.  Fig. \ref{fig:horiz}D outlines these vortices, 
where the area encompassed by a loop corresponds to the strength 
of the associated vortex. The velocity vectors associated with
these vortices have both downstream and lateral components.
Further downstream, the strength of the vorticity decreases, and
the velocity vectors orient more along the direction of the
flow. SM2 shows the time evolution of the velocity vector field
with associated vorticity over one cycle.

\subsubsection{Transverse planes}

Fig \ref{fig:trans} shows four snapshots of the velocity vector
field associated with the five transverse planes T1-T5. As in
Fig. \ref{fig:horiz}, the snapshots split the fin cycle into
quarters. Because there are two undulations along the fin, the fin rays in
T1, T3, and T5 are all roughly in phase with each other while in
antiphase with rays in T2 and T4.  Supplementary video SM3 shows the time evolution of these planes over one cycle.

The velocity vectors in these transverse planes have the
smallest magnitude in the anterior T1 plane, with magnitudes
increasing moving posteriorly to T5. Vortices of alternating
direction are present in each plane upon reversal of the fin
ray, as indicated in Fig. \ref{fig:trans}D. These vortices are weakest in T1 and increase in magnitude
moving posteriorly to T5. Consequently, the flow through these
vortices is strongest in T5, while non-existent in T1. These
flows are oriented mostly downward from the fin, with smaller
lateral components that oscillate from left to right over a
cycle.  Also, vortices remain closely associated with the fin edge in 
T1 as the fin moves laterally, rather than being shed downwards off the fin edge as is evident in T5.

%The vortex structure described is very similar to the reverse
%von K\`arm\`an vortex street that has been described in the wake
%of oscillating foils and the caudal fish of carangiform and
%thunniform swimmers. T5 exhibits the clearest reverse von
%K\`arm\`an vortex street pattern, though it is still present in
%T2-T4. The pattern does not occur in T1, as the tip vortices of
%the fin section in T1 are not shed and are carried laterally
%with the fin. This lateral excursion of vortices is present in
%the other planes as well, after they are shed from the fin at
%varying distances from the fin depending on the position of the
%plane along the length of the fin.

\subsection{Simulation flow structure visualizations}

To visualize the 3D wake structure of the simulations, we visualized
iso-surfaces of the $q$-criterion \citep*{Bora08a}. The $q$-criterion
subtracts the strain rate from the rotation rate of a velocity vector
field, so that regions of positive $q$ indicate vortical structures. Fig.
\ref{fig:isovortcomparison} shows the bottom and side views of these iso-$q$
visualizations for the four simulations. Vortical
structures which are consistently present across the three simulations are indicated by the magenta arrows.
These structures are vortex tubes which are initially shed by the rostral
half-wave of the fin and then continually shed at the peaks of the
waveform as the wave travels caudally. Supplementary videos SM4, SM5, and
SM6 show the evolution of these structures over time. As the shed vortex
tubes are left by the moving fin, they become entangled downstream to form
a complex wake structure, until the strength of the vorticity dies out.

The fourth simulation, labeled FinSim, gives the clearest view of the
wake structure. Video SM7 shows the evolution of the iso-vorticity
structure of the wake, and a snapshot view is included in Fig.
\ref{fig:isovortcomparison}D for comparison to the other simulations.
The flow structures in FinSim are less entangled than those in FishSim1 and FishSim2, 
likely because the traveling wave frequency is less by more than a 
factor of 3 (Table \ref{tab:exp_sets}), giving a clearer view of
the wake. However, the manner in which these vortex tubes are shed are
similar to the vortex tubes from the other simulations.


\section{Discussion}

Both the experimental and computational results indicate that the wake
structure across both fish and robot body and fin morphologies maintain
the same qualitative features. First, we discuss how exactly the flow features of the 2D
PIV planes map to the overall 3D vortical structure of the wake.  
Second, we compare the 3D wake of free-swimming ribbon fins to prior 
results from a non-translating undulating fin \citep*{Shir08a}.
Third, we introduce important relationships between the kinematic
parameters of the undulating fin and non-dimensional numbers associated
with free swimming such as wave efficiency (sometimes referred to as slip)
and Strouhal number. Fourth, we relate these non-dimensional numbers
to the wake structure of the undulating ribbon fin. Last, we compare
the wake of the ribbon fin to the well-characterized wake of carangiform
swimming, showing specifically how knifefish have maintained similar
thrust producing mechanisms with a fin of drastically different morphology
and kinematics.

\subsection{3D schematic of vortex tubes shed from the fin edge}

There has been some effort to show how experimental 2D PIV data correlates 
with hypothesized 3D wake structures in various forms of swimming.
For example, the wake of a carangiform swimmer with an oscillating caudal
tail is thought to generate a series of connected vortex rings \citep*{Naue02b}. A horizontal slice
through these vortex rings illuminates the classic reverse von K\`arm\`an
vortex street visible in planar PIV measurements \citep*{Naue02b}. Similar reverse von
K\`arm\`an vortex streets are present in the transverse planes measured
for the Ghostbot fin, especially in T3-T5 (Fig. \ref{fig:trans}). The vortices in the transverse planes
must be sections of a 3D vortex structure. Indeed, a vortex tube is shed
as the wave travels along the length of the fin, as seen in the
iso-$q$ visualizations of the simulations (SM4, SM5, SM6, SM7, and Fig. \ref{fig:isovortcomparison}).

The vortex tubes shed from the fin edge also account for the other vortices present in the
horizontal (H) and midsagittal (M) planes. In the horizontal plane, strong
vortices are present at the posterior point of the intersection between
the fin and the plane indicated by the solid white line in Fig \ref{fig:horiz}. 
These vortices are lengthwise slices through the vortex tube
as it is being shed, orthogonal to the slices in the
transverse planes. In the midsagittal plane, a series of similarly
rotating vortices are shed in a line at an angle to the fin closely
matching the pitch angle of the robot (Fig. \ref{fig:midsag}). Again, 2D vortices are slices
through the vortex tubes shed off the fin. As seen in Fig.
\ref{fig:isovortcomparison} the vortex tubes bend medially towards the opposing side and cross the midsagittal plane.
Successive vortex tubes have opposite rotation in the transverse plane (Fig. \ref{fig:trans}), but are also shed on
opposite sides of the fin. Therefore, when the vortex tubes bend medially and intersect the midsagittal
plane, every vortex tube has the same rotational direction such that higher
downstream flow (indicated by the light blue color) is closer to the fin. Over the course of a cycle, this
high downstream flow averages to the jet visible in Fig \ref{fig:midsag}C.

The shedding and distortion of the fin edge vortex tubes are schematized in Fig.
\ref{fig:schematic}. Vortex tubes are color-coded based on the side of the
fin from which they are shed. This schematic encapsulates many of the main
features present in the vortex tubes shed by the fin edge. Towards the
rostral portion of the fin, where the tube is just starting to be shed,
the tube is oriented laterally due to being shed at a single time by the
rostral half wave of the fin (Fig. \ref{fig:schematic}A). Each vortex tube should be connected to the
previous tube as well as the subsequent tube according to 
Helmholtz's second theorem of vorticity \citep*{Sche50a}. The schematic
details this connection for the anterior part of these tubes, though they
should also connect on the posterior ends of the tubes after being shed off
the back of the fin. In between, the vortex tube remains roughly in the
location where that portion of the tube was shed (Fig. \ref{fig:schematic}B). Therefore, the
downstream spacing of subsequent vortex tubes is mostly determined by the
swimming speed of the fin. We will return to the issue of vortex spacing
in further discussion below.

The wake structure remains consistent throughout all
simulations, which range in Reynolds number over about one order of
magnitude. Fig. \ref{fig:isovortcomparison} highlights the consistent
vortex structures when viewed from below. Furthermore, Fig.
\ref{fig:mid_compare} compares the midsagittal plane of the schematics with
the PIV data as well as all four simulations. In Fig. \ref{fig:mid_compare}
B-E, the line of blue-colored vortices emanating from the fin along the
direction of swimming is consistent with the lateral reorientation of the
edge vortex tubes. Fig. \ref{fig:mid_compare}E further shows the connection
between the vorticity in the midsagittal plane and the 3D vortex
structure. Whether robot scale, fish scale or without any body, the wake
structure remains qualitatively similar.

\subsection{Comparison of wake structures from free-swimming and non-translating ribbon fins}

The hydrodynamics of elongated ribbon fin propulsion is highly complex
with significant three-dimensional effects. Previously, we investigated
the hydrodynamics of a non-translating undulating ribbon fin in stationary
water \citep*{Shir08a}. This flow condition was used to explore the
hydrodynamics of impulsive motion. Impulsive motion is common in knifefish
as they are constantly changing their direction of motion during prey capture \citep{MacI01a}.

The non-translating undulating fin produced both surge and heave forces,
just as the average velocity of a freely-swimming fin has both surge and
heave components. We attributed the heave forces in the non-translating
case to vortex tubes generated by the fin edge as it oscillates back and
forth laterally. Similarly, Fig. \ref{fig:trans} shows how the vorticity
generated and shed by the fin edge creates a flow with large heave
components. We attributed the surge forces in the non-translating case to
the longitudinal jet created by the motion of the traveling wave. We
inferred that the primary vortex rings encircling the jet were induced by
the flow and were separate from the vortex tubes generated by the fin
edge. However, in the current free-swimming simulations as well as the PIV
data, we see that the vortex tubes shed by the fin become distorted and
actually wrap around the overall jet shown in Fig. \ref{fig:midsag}.
Consequently, the vortex rings and the fin edge vortex tubes that 
were previously thought to be separate phenomena associated 
with surge and heave forces respectively are likely different parts 
of the same vortex structures shed by the fin. A
comparison of these vortex rings of the current FinSim and the
non-translating fin from \citet*{Shir08a} is shown in Fig.
\ref{fig:front_view}.

\begin{comment}
%The main difference between the two cases is whether or not the fin is
%allowed to move away from the flow structures as they are generated. For
%example, we previously inferred that the fin edge vortex tubes stayed
%attached to the fin edge for the non-translating fin, but this is likely
%because the fin stays in the regions where the vortices are generated.
%Similarly, as the shed vortex tubes encircle the jet, they will start to
%accumulate together in the non-translating fin, whereas the vortex tubes remain
%more regularly spaced in the free-swimming case.
%
%
%In this first study, we found that the main
%mechanism of thrust production is the generation of a streamwise central jet with
%associated vortex structures.
%
%For impulsive motion, the vortex structures can be
%broadly identified as 1) vortex rings attached to fin with a crab-shaped structure, 2)
%vortex tubes attached to the tip of the ribbon fin, and 3) secondary vortex structures
%that might be generated by the interaction of the other vortex structures with the
%surrounding fluid. We associated the vortex rings attached to the fin with the generation of surge force
%while the vortex tubes attached to the fin tip with the generation of heave force.
%
%In this current work, we considered the hydrodynamics of a freely swimming ribbon
%fin. Similar to the impulsive motion situation, the thrust mechanism is associated 
%with a clear
%jet generated by the ribbon fin (Fig. \ref{fig:midsag}B). However, this jet is not parallel the longitudinal axis
%of the fin, it is inclined. For the parameters and the geometry of the fin considered,
%the angle between the fin axis and the ribbon fin jet is approximately 13 degrees. 
%The predominant vortex structure observed is a series of
%inter-connected vortex tubes that originate at the fin tip and are shed as
%the traveling wave progress. Similar to the jet generated by the ribbon fin, the
%vortex structures are shed at an angle with respect to the long axis of the ribbon fin.
%These vortex structures resemble the vortex tubes previously described for impulsive
%motion, with the main difference that during free swimming these vortex structure
%do not stay attached to the fin but are continuously shed as the traveling wave
%propagates. 
%%We can speculate that this shedding (or attachment to the fin) is highly
%%dependent to the Strouhal number (i.e. the ratio between lateral fin velocity to
%%the swimming speed). 
%Another difference between the impulsive motion vs free- swimming motion
%is the apparent lack of vortex rings around the ribbon fin. However,
%vortex tubes shed from the fin edge in the free-swimming case do seem to
%curl up around the other side of the fin, and it is possible that the
%vortex rings around the fin described in \citet*{Shir08a} might have a
%similar origin. To fully compare impulsive to free-swimming flow
%structures, further simulations and experiments are necessary to control
%for other factors such as fin kinematic parameters and non-dimensional
%constants such as Reynolds number and Strouhal number, which are beyond the scope of this article.
%%It is possible
%%that the angle of the central jet with respect to the fin axis results in a strong
%%interaction with the vortex tubes associated the fin tip, which results in just one
%%common vortex structure. However, this interaction it is not clear with this current
%%work. Additional studies with different Strouhal number will be needed to clarify
%%this.
\end{comment}

\subsection{Relationship between Strouhal number, wave efficiency, and fin parameters}

For any undulatory swimmer in which a traveling wave is present, a wave
efficiency $\eta$, can be defined as
\begin{equation} \eta = \frac{U}{V_\text{wave}} \label{eq:wave_efficiency}
\end{equation}
where $U$ is the free-swimming speed and $V_\text{wave} = f \lambda$ is the
speed of the traveling wave. $\eta$ has a range of $0 - 1$, where 0
indicates no longitudinal movement, and 1 indicates that the fin is moving
forward as fast as the traveling wave moves backward. $\eta$ therefore is
labeled as a measure of efficiency, as for the best case scenario where
$\eta = 1$, the fish achieves the highest output velocity for the
velocity given to the traveling wave. In our
simulations and robotic experiment, the wave efficiency ranges from $0.3$
to $0.55$ (Table \ref{tab:exp_sets}), whereas \albif{} has been recorded to
achieve wave efficiencies as high as $0.7$ \citep*{Ruiz13a}.

The Strouhal number (St) has been cited as an important non-dimensional
number for unsteady, periodic flows such as those found in the wake of
fishes \citep{Triantafyllou91,Laud06c}. The equation for St is
\begin{equation} St = \frac{f A}{U} \label{eq:strouhal}
\end{equation}
where $f$ is the frequency of ray oscillation, $A$ is a length associated
with the oscillation (defined in this work as the peak to peak
lateral excursion of the fin), and $U$ is the free-swimming speed. The
range of St for cruising swimmers and fliers of various sizes is between
$0.2$ and $0.4$ \citep*{Tayl03a}. The simulations and free-swimming robotic data
presented here exhibit St between 0.3 and 0.65 (Table
\ref{tab:exp_sets}).

St can be considered to indicate the ratio of local inertial forces to
convective inertial forces, or in the case of a swimming fish, the ratio
of local momentum input to the wake by the fin to the output momentum
imparted on the fish by the wake which moves the fish forward. Therefore,
St can be considered a second measure of efficiency, where the fin aims to
achieve the highest swimming velocity while minimizing lateral
fin velocity. When St is considered as a type of efficiency, clearly a
lower St implies better performance, with $St = 0$ perfect performance.
The natural range of $0.2$ to $0.4$ for St indicates that there exists a
practical lower limit of around 0.2 (gliding is a case where $St = 0$,
though the gliding velocity is not constant over time).

As $\eta$ and St are indirect indicators of efficiency, how
do these two parameters depend on each other? Multiplying Eqn.
\ref{eq:wave_efficiency} (with $f \lambda$ substituted for $V_\text{wave}$) by
Eqn. \ref{eq:strouhal} results in
\begin{equation} \eta St = \frac{A}{\lambda} \label{eq:specific_amp}
\end{equation}
where the quantity $\frac{A}{\lambda}$ is termed the specific amplitude (in this case the peak-to-peak amplitude).
This relationship implies that for a given specific
amplitude, there is a lower limit for St for the maximum value of $\eta =
1$. Therefore, to decrease the lower limit of St, the specific amplitude
of the traveling wave must be diminished, either by decreasing $A$ or
increasing $\lambda$. However, assuming there exists the physical lower
limit to St of 0.2, decreasing the specific amplitude also decreases the
maximum achievable $\eta$. For example, if the highest achievable $\eta$
was 0.5 and the lowest achievable St was 0.2, the specific amplitude would
necessarily be 0.1.

%The specific amplitude for the robot given the nominal fin kinematic
%parameters used in this work was 0.15, and the robot
%achieved a wave efficiency and St of 0.53 and 0.29 respectively.

It should be noted that a better measure of efficiency for a cruising
animal or robot would relate the total input power ($P$) to the system to
the useful output power, such as propulsive efficiency ($\eta_P
=\frac{FU}{P}$) or cost of transport ($COT = \frac{P}{U}$). For example,
Anderson et al. found that $\eta_P$ was maximal for St in the range of
0.25 to 0.40 for an oscillating foil with varying kinematics
\citep*{Ande98a}. Moreover, Moored et al. found a local maximum of $\eta_P$
when the driving frequency of an oscillating fin matched the hydrodynamic
resonant frequency \citep{Moor12a}. However, these oscillating foils were
not self-propelled. While we did not measure input power to the system, we
would expect the power needed to drive the rays of either the robot or the
fish to increase with ray velocity, which increases with both frequency
and amplitude of oscillation, parameters in the numerator of St.
Therefore, we expect COT to follow similar trends as St, and that the
lowest St observed are approaching the best propulsive efficiencies
achievable.

\subsection{Wake structure in terms of St and $\eta$}

More pertinent to the focus of this study is the relationship of St and
$\eta$ to the actual wake structure generated by the fin.
In these free-swimming cases, the vortex tubes shed off the fin exhibit a 
small downstream velocity component relative to the surrounding fluid.
Therefore, the spacing of adjacent vortex tubes in the direction of
swimming is mostly determined by the swimming speed, $U$. Because two
vortex tubes are shed for every ray cycle, the vortex spacing $S$ along the direction of motion is approximately
\begin{equation} S = \frac{U}{2 f} \label{eq:spacing}.
\end{equation}
Substituting Eqn. \ref{eq:spacing} into Eqn. \ref{eq:wave_efficiency} and
solving for S gives
\begin{equation} S = \frac{\eta \lambda}{2} \label{eq:spacing_eta},
\end{equation}
and similarly substituting Eqn. \ref{eq:spacing} into Eqn.
\ref{eq:strouhal} and solving for S gives
\begin{equation} S = \frac{A}{2 St} \label{eq:spacing_st}.
\end{equation}
From both of these relationships, it is clear that higher wave
efficiencies and lower Strouhal numbers result in larger vortex spacing.
Higher vortex spacing indicates that the fin is obtaining more impulse per
vortex tube, which would indicate greater efficiency. It is hypothesized from
the well studied carangiform wake, however, that a vortex spacing resulting in
a reverse Von K\`arm\`an street with associated linked small core vortex
rings is the signature of highly propulsive thrust \citep*{Naue02b}. These
linked rings seem to limit the vortex spacing, which in turn limits St.
For example, when St $=0.2$, $S = 2.5A$ so for vortex spacing greater than
this value the linked vortex pattern likely becomes unlinked and breaks down.
Further discussion on predicting St from measured distances associated 
with the wake for ribbon fins as well as other forms of swimming is provided in a later section.

There have been a few studies relating the wake structures to St in both
anguilliform and carangiform forms of swimming. A series of articles by
Borazjani and Sotiropoulos show in simulation that as Reynolds number
increases, St decreases for free swimming carangiform and anguilliform
swimming \citep*{Bora08a,Bora09a}. For low Reynolds numbers (Re = 300 and
4000), St was always around 0.6 or above. The St fell within the observed
range of 0.2 to 0.4 only for the inviscid simulations. For these low St
scenarios, the wake consisted of a single row wake consisted with the
linked vortex rings associated with carangiform swimming. For higher St,
the wake split into a double row wake structure. In self-propelled cases,
the transition from single row to double row wake structure seems to occur
as the vortex spacing decreases past a critical value, corresponding with
increasing St. We expect that if the ribbon fin swam at higher St, a
similar double row wake structure would develop. Other studies of heaving
and pitching foils in an imposed flow show that changes in effective angle
of attack also can cause the transition from single row to double row wake
structure \citep*{Blon05a, Dewe12a}.

For undulatory fins such as the knifefish fin, the vortex spacing also
depends on wavelength and wave efficiency given by Eqn.
\ref{eq:spacing_eta}. This trend is visible in Fig. \ref{fig:mid_compare}B,
where the spacing of the line of vortices in the midsagittal plane is
determined by the wave efficiency, as the wavelength is scaled in each
image to be equal. For a case with high wave efficiency, such as FinSim
(Fig. \ref{fig:mid_compare}E), the spacing is visibly larger than a case
with low wave efficiency, such as FishSim1 (Fig. \ref{fig:mid_compare}C). 
Clearly, if wavelength were
further decreased, the vortex spacing would decrease, resulting in
interference between adjacent vortex tubes. However, if wavelength were
increased, the surge component of the thrust would vanish, as described in
the following section.


\subsection{Connecting the wake of the ribbon fin to that of carangiform swimming}

The feature of the ribbon fin that differentiates it from most other forms of
underwater propulsion is that the thrust produced has components in both
the surge and heave axes. For that reason, there exists a
consistent insertion angle of the fin to the body, so that the
longitudinal axis of the body aligns more with the direction of thrust and the corresponding wake.
The ribbon fin can even vary the proportion of the surge component of
thrust to the heave component of thrust, though fish accomplish this
mainly using two counter-propagating waves \citep*{Cure11b,Sefa12a}. Can the
flow structure for a single traveling wave be broken into heave features
and surge features? Consider the wake generated by an oscillating caudal
fin of a carangiform swimmer such as a trout, as shown in Fig.
\ref{fig:vortexrings}A. As mentioned, the wake consists of a series of
linked vortex rings, with an increased flow through those rings. There are two
vortex rings per cycle, and each associated jet has a lateral component
with alternating direction. The height of the ring is on the order of the
height of the fin, while the axial spacing of the rings is determined mostly by
the Strouhal number and amplitude of oscillation.

The structures created by a trout surging can be related to the
structures generated by a ribbon fin if it were to be actuated with a traveling
wave of infinite wavelength. In this
case, as shown in Fig. \ref{fig:vortexrings}B, each ray of the fin is in
phase with every other ray, so the fin oscillates similarly to the caudal
fin of a trout. This oscillation is now centered around the heave axis of
the ribbon fin instead of the surge axis of the caudal fin, and would tend
to move the fish in the fin-fixed heave direction. Assuming an appreciable
velocity could be reached so that St fell within the range of
approximately 0.2 to 0.4, we predict that the wake would contain a similar trail of linked
vortex rings, now with the width of the rings on the order of the length
of the fin as shown in Fig. \ref{fig:vortexrings}B.

Moving to the biological case of an undulating fin with a finite
wavelength, as shown in Fig. \ref{fig:vortexrings}C, the portion of the vortex tubes
which bend up from the rostrocaudally stretched tube are no longer shed
simultaneously, but over a period of time as the wave travels along the
fin. This wave motion in the surge direction contributes momentum to the
fluid along that axis so that a surge component of thrust emerges. The
vortex rings are now realigned along a line at an angle ventral to the
fin, approximately the same angle as the insertion angle of the fin to the
body. The eel-like undulatory motion of the ribbon fin is responsible for
this realignment of the wake structures and therefore the off-axis
direction of thrust. However, the vortex ring pattern shares more
commonality with the small core linked vortex rings found in the wake of
carangiform swimmers, rather than the double row wake structures found in
anguilliform locomotion \citep*{Tyte04a,Kern06a}.

\subsection{The predictive power of a ribbon fin 'footprint'}

The wake of a fish can be considered the `footprint' of the fish. We
have described the footprint of an undulating fin and explored the
similarities and differences compared to the footprints of other forms of
underwater locomotion. The larger structure of the wake of a self-propelled fish can predict
the swimming mode. First, a wake with double row vortex rings and
associated lateral jets implies an anguilliform swimmer
\citep*{Tyte04a,Kern06a}. Second, a wake with a single row of linked
vortex rings and associated downstream jet implies a carangiform swimmer
\citep*{Naue02b}. Third, a wake with a single row of diagonally distorted and linked
vortex tubes implies a undulating ribbon fin swimmer. It is important to
note that the transition from a single row wake to a bifurcated double row
wake is due mostly to the differences in St exhibited by anguilliform and
carangiform swimmers, not simply due to differences in kinematics
\citep*{Tyte10a}, leading us to predict that if a ribbon fin fish swam at
a higher St, the wake would begin to bifurcate.

Furthermore, a closer examination of the vortex structure can produce an
estimation of St. Using the equation for the vortex spacing along the
swimming direction in terms of St (Eqn. \ref{eq:spacing_eta}), we can
predict St by measuring the distance between successive vortex tubes or
rings as well as the width of the wake (which corresponds to $A$ in 
Eqn. \ref{eq:spacing_eta}). The lower the ratio of the wake width to the vortex spacing,
the lower the St, indicating a fish that is able to swim further per
`footprint.' These predictions can be extended to the bifurcating wake of
an eel, where the distance $S$ is the streamwise distance between two
alternating vortex rings, which is indeed much smaller than that of a
carangiform swimmer, hence the difference in St.




\noindent
\textbf{Acknowledgment}\\

M.A.M. led the experimental fluids and robotics efforts.  I.D.N. and O.M.C. developed the robotic experiments.  I.D.N ran and analyzed the PIV experiments. N.A.P. led the computational fluid analysis effort.  A.P.S.B wrote the computational fluid dynamics code.  R.B. ran and visualized the simulations.   I.D.N. wrote the manuscript with contributions from A.P.S.B, R.B.,  O.M.C., and M.A.M.

We thank George Lauder for his helpful discussions concerning fish swimming mechanics and PIV analysis.  We also thank James Snyder for his help in programming the Ghostbot as well as Stuart Cameron for his advice on the design of the PIV system.

This work was supported by NSF grant IOB-0517683 to M.A.M. and by an NSF CDI grant CMMI-0941674 to M.A.M, and N.A.P. Partial support was provided by NSF grants CBET-1066575 to N.A.P and CBET-0828749 to N.A.P. and M.A.M. O.M.C acknowledges the support of a D.F.I fellowship.  Computational resources were provided by Northwestern University High Performance Computing System -- Quest



\bibliographystyle{jexpbiol}  
\bibliography{../../abbreviated_journal_titles,../../references}

%\bibliographystyle{plos}
\noindent
{\bf Electronic Supplementary Material}\\


%%%% Figures
%Figs intro and material and methods

\newpage
\begin{table}
  \centering
  \begin{tabular}{ |c|c|c|c|c|c|}
    \hline
		
		& \emph{FishSim1}
		& \emph{FishSim2}
		& \emph{GhostbotSim}
		& \emph{Ghostbot}
		& \emph{FinSim}\\
   \hline
   \hline
   $f$ (Hz) & 10 & \cellcolor{red!25} 10 & \cellcolor{red!25}  3  & 3 & 3\\
   $L_\text{fin}/\lambda$ & 2.5 & \cellcolor{red!25} 2.5 & \cellcolor{red!25} 2 & 2 & 2\\
   $\theta_\text{max}$ (deg) & \cellcolor{blue!25} 23 & \cellcolor{blue!25} 30 & 30 & 30 & 30\\
   A$/\lambda$ & \cellcolor{blue!25} 0.22 & \cellcolor{blue!25} 0.28 & 0.31 & 0.31 & 0.20\\
   Pitch angle (deg) & 10 & 10 & 8 & 13 & 9\\
   U (cm$/$s, $L_\text{fin}/$s) & 12, 1.33 & 17, 1.93 & 26.5, 0.81 & 26, 0.80 & 9, 0.89\\
   $\eta$ & 0.33 & 0.48 & 0.54 & 0.53 & 0.59\\
   St & .65 & 0.58 & 0.57 & 0.58 & 0.34\\
   Re & 13000 & 13000 & 80000 & 80000 & 8000\\
    \hline
  \end{tabular}
  \caption{Fin parameters and free-swimming results}
  \label{tab:exp_sets}
\end{table}


\newpage
\begin{table}
 \centering
  \begin{tabular}{|c|c|c|c|c|}
    \hline
		& \emph{FinSim}
		& \emph{FishSim1}
		& \emph{FishSim2}
		& \emph{GhostbotSim}\\
   \hline
   \hline
  Domain Size (cm$^3$) & 72 $\times$ 10 $\times$ 24 & 100 $\times$ 18 $\times$ 30 & 60 $\times$ 11 $\times$ 12 & 260 $\times$ 40 $\times$ 60 \\
  Refinement ratio  & 4  & 4 & 4 & 4\\
  No. grid levels   & 4  & 4 & 3 & 3 \\
  Grid size at coarsest level & 96 $\times$ 10 $\times$ 54  & 144 $\times$ 24 $\times$ 24 & 128 $\times$ 24 $\times$ 26 & 196 $\times$ 30 $\times$ 45  \\
  Vorticity tagging threshold (s$^{-1}$)& 40 & 100 & 80 & 26 \\ 
 \hline    
  \end{tabular}
\caption{Numerical simulation parameters}
 \label{tab:sim_par}
 
\end{table}

\section*{Table Legends}
Table~\ref{tab:exp_sets}\\
The blue cells highlight the fin parameters that were changed in the transition from FishSim1 to FishSim2.  Note that the amplitude profile also change from a variable profile based off of actual fish data in FishSim1 to a constant profile in FishSim2.  The red cells highlight the fin parameters that were changed in the transition from FishSim2 to GhostbotSim.  Note also that the scale of the GhostbotSim was roughly three times the size of FishSim1 and FishSim2.





\newpage \begin{figure} \centering
\includegraphics[width=80mm]{Fig01_fishandrobot.pdf} \caption{The model
system and its physical model. 
(A) The weakly electric black ghost
knifefish, \emph{Apteronotus albifrons}, a gymnotiform swimmer from the
rivers of South America that undulates its elongated ventral fin, commonly
termed the ribbon fin, to propel itself through the fluid. Photograph
courtesy of Per Erik Sviland. 
(B) The ``Ghostbot'' swims using a
biomimetic ribbon fin (black). The fin is 32~cm long and is actuated by 32
independent fin-rays. 
(C) Schematic of the ribbon fin showing the
wavelength ($\lambda$), the angular fin amplitude ($\theta$), ray
oscillation frequency ($f$), and the Eulerian reference frame and the
robot body frame (\emph{surge}, \emph{heave}, and \emph{sway}).}
\label{fig:fish_robot} \end{figure} \clearpage \newpage

\begin{figure}
\centering
\includegraphics[width=80mm]{Fig02_experimentalsetup.pdf}
\caption{Schematic of experimental setup. 
(A) The robot was suspended into a variable speed flow tank from a
frictionless air bearing platform. During PIV experiments, the robot was
fixed in place so that it would remain stationary in the laser plane, and
the flow speed was set to match the previously determined free-swimming
velocity for the robot with a pitch angle of $13^\circ$, frequency $f$ of
3~Hz, wavelength $\lambda$ of $L_\text{fin}/2$, and angular amplitude $\theta_\text{max}$
of $30^\circ$. 
(B) 7 PIV planes were collected. The midsagittal plane (M)
actually consists of 3 sets of PIV data that were reassembled. The
horizontal plane (H) conststs of 4 sets of PIV data, and intersects the
fin edge approximately halfway along the length of the fin. Each of the transverse
planes (T1-T5) consisted of a single set of data, and were
approximately located at each quarter length of the fin length $L_\text{fin}$.}
\label{fig:exp_setup}
\end{figure}
\clearpage
\newpage

\begin{figure}
\centering
\includegraphics[width=80mm]{Fig03_midsag3dvort.pdf}
\caption{Flow features in the midsagittal plane (plane M from Fig. \ref{fig:exp_setup}B). 
(A) shows an instantaneous snapshot of the velocity vector field in the
midsagittal plane with the mean swimming velocity ($U = 26$~cm/s) 
subtracted. The colormap indicates the vorticity. A line of
clockwise vortices emanate from near the front of the fin at an angle to
the fin close to the pitch angle of the robot. See SM1 for the corresponding movie.
(B) shows the phase-averaged velocity vector field and corresponding vorticity at the same time in the cycle as (A).
(C) shows the average jet
generated by the traveling wave of the fin. The colormap indicates the
magnitude of the fluid velocity field averaged over many cycles. The jet
is aligned with the swimming direction, and is strongest immediately
downstream of the fin.}
\label{fig:midsag}
\end{figure}
\clearpage
\newpage

\begin{figure}
\centering
\includegraphics[width=80mm]{Fig04_horiz3dvort_bottomview.pdf}
\caption{Flow features in the horizontal plane (plane H from Fig. \ref{fig:exp_setup}B). 
Velocity vector fields were measured in a plane horizontal with respect to
the wake structure (see inset for plane position and viewpoint). The vertical position
of this plane was selected to bisect the wake. (A-D) show four phase-averaged snapshots
within one cycle at $t = \frac{1}{4}T$ (A), $t = \frac{1}{2} T$ (B), $t =
\frac{3}{4}T$ (C), $t = T$ (D).  The vortices are outlined with 
white arrows in (D) to show the sense of rotation as well as 
relative magnitude of the vorticity which is indicated by the size of the closed loops.  
The movement of the traveling wave is indicated by the transparent white strip.  
See SM2 for the corresponding movie.}
\label{fig:horiz}
\end{figure}
\clearpage
\newpage

\begin{figure}
\centering
\includegraphics[width=120mm]{Fig05_trans3dvort.pdf}
\caption{Flow features in the transverse planes (planes T1-T5 from Fig. \ref{fig:exp_setup}B). 
Phased-averaged velocity vector fields for the five transverse planes are shown as they
correspond actual fin motion. (A-D) show four
snapshots within one cycle as in Fig. \ref{fig:horiz}. Plane T5 at the
downstream edge of the fin indicates that counter-rotating vortices are
shed each time the fin changes its lateral direction. There is an
associated jet of fluid through these vortices with a strong downward
component of the fin edge. This flow pattern weakens for the more upstream
planes, to the point where in plane T1 the vortices are not shed and
there is little downward flow off the fin edge.  
A normal view of plane T5 is included to give a clear picture of the reverse 
Von K\`arm\`an vortex street, which is specifically outlined in plane T5 of (D).
See SM3 for the corresponding movie.
 }
\label{fig:trans}
\end{figure}
\clearpage
\newpage

\begin{figure}
\centering
\includegraphics[width=160mm]{Fig06_IsoVortComparison.pdf}
\caption{3D flow structure comparison.
Bottom and side views of the 3D flow structure is shown for the four simulations:
FishSim1 (A), RoboSim (B), FishSim2 (C), and FinSim (D). The 
structures are an iso-surface of the $q$-criterion and are colored such
that dark-blue indicates upstream flow velocities and light-blue indicates
downstream flow velocities. The fin edge is outlined with a dashed yellow
line for the bottom views. Vortex tubes are indicated by the pink arrows
at the points where they cross laterally over the midsagittal plane.  
These vortex tubes share many qualitative similarities across all four simulations, 
including how they are shed from the fin edge as well as how they bend laterally to connect to one another.  The `L' in (A) indicates the anatomical left side of the fish for both the side and bottom views, which remains consitent throughout (B-D)
 }
\label{fig:isovortcomparison}
\end{figure}
\clearpage
\newpage

\begin{figure}
\centering
\includegraphics[width=80mm]{Fig07_schematic_3view.pdf}
\caption{ Schematic of 3D vortex structure.
(A) Vortex tube as it begins to shed from the fin. In this schematic,
locations of the waveform peaks over time are indicated with X's and O's,
where the solid black symbols indicate the present location of the peaks
while faded symbols indicate previous locations. The vortex tube
associated with the X's just begins to shed as the subsequent waveform
peak represented by the O's appears on the fin. 
(B) The later position of
the fin after the waveform peak represented by the X's reaches the back of
the fin. The original position of the fin from (A) is included as a faded
silhouette. The blue vortex tube associated with the X peak joins with the
subsequent vortex tube associated with the O peak in approximately the
same location as the fin in (A). Otherwise, the vortex tubes follow the
locations where the peaks of the waveform transversed. Only two vortex
tubes and corresponding peaks are shown for clarity.
(C) Snapshot of multiple successive vortex tubes. Each vortex tube is shed
as the corresponding peak of the waveform travels from the front to the
back of the fin. The vortex tubes have been colored to indicate whether
they were shed from the left (red) or right (blue) side of the fin. Each
vortex tube is directly laterally to join with the proceeding tube as in
(B). Similar linking is expected further downstream in the wake as the
tubes are shed off the rear of the fin. The expanded view of the vortex
tubes shows the sense of rotation with circular arrows. The two arrows
marked (a) indicate opposite directions of rotation of the red and blue
tubes in the orthogonal transverse plane, while the two arrows marked (b)
indicate the same direction of counter-clockwise rotation of two tubes as they cross the
midsagittal plane.  These midsagittal vortex tube sections the same 
as the horizontal line of blue vortices from the 2D PIV data from Fig. \ref{fig:midsag}A.
 }
\label{fig:schematic}
\end{figure}
\clearpage
\newpage

\begin{figure}
\centering
\includegraphics[width=80mm]{Fig08_midsag_compare.pdf}
\caption{ Comparison of wake structure in the midsagittal plane.
(A) Midsagittal view of schematic.  The schematic from Fig \ref{fig:schematic}C with a transparent gray
midsagittal plane shows where each vortex tube crosses the plane.
Consecutive vortex tubes are shed from opposite sides of the fin, as
indicated by the color of the tubes. Each vortex tube has the
same sense of vorticity when crossing through the midsagittal plane as described in Fig. \ref{fig:schematic}C. The
scale bar shows the wavelength, where each image in (A-E) has been scaled
so that wavelength is consistently sized.
(B) Midsagittal view of PIV data. The midsagittal plane from Fig. \ref{fig:midsag}B is shown again for
comparison. The line of blue vortices correspond with the traversals of
the shed vortex tubes across the midsagittal plane.
(C) Midsagittal view of RoboSim. This view shows a similar line of
vortices as (B), marked by magenta arrows for clarity.
(D) Midsagittal view of FishSim1. The line of vortices, indicated by 
the magenta arrows for clarity, is still present, though the vortices are
spaced more closely together. The spacing of the vortices is dependent on
wavelength and wave efficiency (Eq. \ref{eq:spacing}). As wavelength is scaled
the same across each case, a closer spacing indicates a smaller wave
efficiency, which is the case for FishSim1 (see Table \ref{tab:exp_sets}).
(E) Midsagittal view of FinSim. The vorticity for the midsagittal plane of FinSim is shown with the 3D
iso-vorticity structure (transparent) to show the correlation between the
planar vorticity and the overall wake structure. In this case, the
vortices are spaced further apart as the wave efficiency is highest.
 }
\label{fig:mid_compare}
\end{figure}
\clearpage
\newpage

\begin{figure}
\centering
\includegraphics[width=80mm]{Fig08a_shir08_compare.pdf}
\caption{ Comparison between free-swimming and non-translating ribbon fins.
(A) Front view of the vortex structure from a non-translating fin \citep*{Shir08a}. In
this previous study, we investigated the hydrodynamics of impulsive motion
through simulations of a non-translating undulating fin. Vortex rings
surrounded the jet created by the traveling wave along the fin. The
iso-vorticity surface was colored by the axial flow velocity, where red
indicates downstream flow velocities and blue indicates upstream flow
velocities. Therefore, the red interior of these rings indicate a
propulsive jet.
(B) Front view of the vortex structure of the freely swimming FinSim. Vortex rings 
similar to those in (A) wrap around the fin, here visualized with 
iso-vorticity surfaces and colored similary as (A). These vortex rings clearly originate
from the fin edge, and are distorted to circle the fin, leading us to
conclude that the vortex rings from the previous non-translating
simulations originate from the fin edge as well. The main difference
between the two cases is that the vortex rings in the non-translating case
become more closely spaced as the fin is not able to move away from the
regions where the rings are shed.
 }
\label{fig:front_view}
\end{figure}
\clearpage
\newpage

\begin{figure}
\centering
\includegraphics[width=160mm]{Fig09_vortexrings.pdf}
\caption{ Comparison of wake stuctures across swimming modes.
(A) A carangiform swimmer such as a trout creates a wake consisting of
linked vortex rings when cruising.
(B) In the hypothetical scenario in which the ribbon fin of a knifefish
oscillates with no traveling wave, the wake structure might bear
similarities to the carangiform wake if the knifefish could swim in the
heave direction at similar St numbers. The vortex rings become elongated,
matching the elongation of the anal fin.
(C) As the ribbon fin transitions from purely oscillatory to undulatory
kinematics with a wavelength smaller than the fin length, the vortex rings
become reoriented due to the motion in the surge direction, matching the
vortex structure from the 3D schematic in Fig. \ref{fig:schematic}. If the
amplitude of the fin motion was the same for both the carangiform swimmer
in (A) and the gymnotiform swimmer in (C), the spacing of the vortex rings
in the swimming direction would depend on St as indicated by Eqn.
\ref{eq:spacing_st}.
 }
\label{fig:vortexrings}
\end{figure}
\clearpage
\newpage


\end{document}




















