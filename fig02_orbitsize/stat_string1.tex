 The finned tetrapods had orbits that were $15 \pm 9.3$~mm long (all numbers mean $\pm$ standard deviation), while the digited tetrapods had orbits that were 33 $\pm$ 20~mm long. The ratio of orbit length to skull length was 13 $\pm$ 4.1\% ($N=21$) for the finned tetrapods, and 20 $\pm$  6.6\% ($N=38$) for the digited tetrapods. We further subdivided the digited tetrapods into two groups: one consisting exclusively of the secondarily-aquatic colosteid-adelospondyl clade ($N=5$), and the digited tetrapods with these taxa removed ($N=33$). The mean orbit length for the secondarily-aquatic digited tetrapods was 15 $\pm$ 9.3~mm long, while the length for the digited tetrapods without the colosteid-adelospondyls was 35 $\pm$ 20~mm. Similarly the ratio of orbit length to skull length for the colosteid-adelospondyls and the digited tetrapods minus this group was 16 $\pm$ 2.8\%, and 21 $\pm$ 6.8\%, respectively.