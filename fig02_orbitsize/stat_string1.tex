 The tetrapodomorph fish had orbits that were $15 \pm 9.3$~mm long (all numbers mean $\pm$ standard deviation), while the stem tetrapods had orbits that were $33 \pm 22$~mm long. The ratio of orbit length to skull length was $0.131 \pm 0.04$  ($N=21$) for the tetrapodomorph fish, and $0.214 \pm  0.06$ ($N=31$) for the stem tetrapods. A two-tailed  T-test (assuming unequal variance) rejected the null hypothesis that the means come from the same distribution ($p = 0.00000016$). A Wilcoxon rank sum test also rejects the null hypothesis ($p = 0.0000016$), as did a two-sample Kolmogorov-Smirnov test ($p = 0.000012$). 