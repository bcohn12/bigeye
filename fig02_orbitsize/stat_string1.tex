The ratio was $0.131 \pm 0.04$  ($N=21$) (all numbers mean $\pm$ standard deviation) for the Tetrapodomorph Fish, and $0.214 \pm  0.06$ ($N=31$) for the Stem Tetrapods. A two-tailed  T-test (assuming unequal variance) rejected the null hypothesis that the means come from the same distribution ($p = 0.00000016$). A Wilcoxon rank sum test also rejects the null hypothesis ($p = 0.0000016$), as did a two-sample Kolmogorov-Smirnov test ($p = 0.000012$).  The absolute size of orbits were also quite different between the groups. The tetrapodomorph fish had orbits that were $15 \pm 9.3$~mm long,while the stem tetrapods had orbits that were $33 \pm 22$~mm long.