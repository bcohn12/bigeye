 The finned tetrapods had orbits that were $15 \pm 9.3$~mm long (all numbers mean $\pm$ standard deviation), while the digited tetrapods had orbits that were 33 $\pm$ 22~mm long. The ratio of orbit length to skull length was 13 $\pm$ 4\% ($N=20$) for the finned tetrapods, and 21 $\pm$  5.3\% ($N=30$) for the digited tetrapods. We further subdivided the digited tetrapods into two groups: one consisting exclusively of the secondarily-aquatic colosteids ($N=5$), and the digited tetrapods with these taxa removed ($N=25$). The mean orbit length for the secondarily-aquatic digited tetrapods was 15 $\pm$ 9.3~mm long, while the length for the digited tetrapods without the colosteids was 37 $\pm$ 22~mm. Similarly the ratio of orbit length to skull length for the colosteids and terrestrial digited tetrapods was 16 $\pm$ 5.2\%, and 22 $\pm$ 5.2\%, respectively.